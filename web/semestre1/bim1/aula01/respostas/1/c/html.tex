Quais são as vantagens, desvantagens e necessidades de uso para HTML:
R:
Vantagens:
  -Fácil de aprender: o HTML é uma linguagem de marcação relativamente fácil de aprender, o que permite que até mesmo iniciantes criem páginas da web.
  -Compatibilidade com navegadores: o HTML é suportado por todos os principais navegadores da web, permitindo que suas páginas da web sejam visualizadas por um grande público.
  -Acessibilidade: o HTML permite que desenvolvedores criem páginas da web acessíveis que possam ser usadas por pessoas com diferentes necessidades e deficiências.
Desvantagens:
  -Limitado em termos de design: o HTML sozinho é limitado em termos de recursos de design, como animações e interatividade. Isso geralmente requer o uso de CSS e JavaScript para adicionar esses recursos.
  -Pode ser complexo: o HTML pode se tornar complexo ao criar páginas da web mais avançadas, especialmente quando combinado com CSS e JavaScript.
  -Vulnerabilidades de segurança: o HTML pode ser vulnerável a ataques de segurança, como injeção de código malicioso (por exemplo, através de scripts).
Necessidades de uso:
  -Criação de páginas da web: o HTML é a base da criação de páginas da web e é essencial para qualquer desenvolvedor web.
  -Compreensão do funcionamento da web: mesmo para aqueles que não desejam se tornar desenvolvedores, é importante ter uma compreensão básica do HTML para entender como a web funciona.
  -Integração com outras tecnologias: o HTML é frequentemente combinado com CSS e JavaScript para criar páginas da web interativas e visualmente atraentes.