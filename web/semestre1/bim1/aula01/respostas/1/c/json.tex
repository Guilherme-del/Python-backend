Quais são as vantagens, desvantagens e necessidades de uso para JSON:
R:
Vantagens:
  -Leveza: o JSON é muito leve em comparação com outras alternativas de intercâmbio de dados, como XML. Ele usa menos largura de banda e é mais rápido para processar.
  -Fácil de ler e escrever: o JSON usa uma sintaxe simples e fácil de ler e escrever. Isso facilita a compreensão dos dados e a criação de APIs e serviços baseados em JSON.
  -Compatibilidade: o JSON é compatível com muitas linguagens de programação e pode ser facilmente processado em diferentes plataformas.
Desvantagens:
  -Limitações: o JSON é uma linguagem de dados simples e não é adequado para representar dados complexos ou documentos estruturados, como o XML.
  -Falta de suporte para comentários: ao contrário do XML, o JSON não oferece suporte para comentários em seus dados.
  -Ausência de especificação formal: embora haja uma especificação para o formato JSON, não há uma especificação formal para o uso de JSON em todas as aplicações.
Necessidades de uso:
  -Troca de dados na web: o JSON é frequentemente usado para troca de dados entre clientes e servidores em aplicativos da web.
  -APIs RESTful: o JSON é uma escolha popular para APIs RESTful, pois é fácil de usar e fornece suporte para muitos recursos importantes, como aninhamento de objetos e listas.
  -Armazenamento de dados: o JSON é adequado para armazenar dados em bancos de dados NoSQL e em serviços de armazenamento em nuvem.