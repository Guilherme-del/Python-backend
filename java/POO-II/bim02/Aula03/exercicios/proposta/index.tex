Exercício 1) Agora em JAVA, crie uma classe chamada Produto que é representada pelos seguintes atributos:
- Código
- Nome
- Volume
- Peso
- Valor

A classe deve disponibilizar os seguintes métodos:
    construtor;
    sets e gets necessários;

Além disso, crie uma segunda classe chamada CestaProdutos que é representada pelos seguintes atributos:
- Nome do cliente;
- ArrayList de Produto;
- Valor total da cesta;

A classe deve disponibilizar os seguintes métodos:
- Consultar valor total;
- Inserir produtos;
- Remover produtos;
- Consultar produto;
- Editar produto;

Construa um menu completo no método Main para utilizar esse classe. Faça com que sua aplicação execute em um laço infinito, sendo que o encerramento deverá ser escolha do usuário.

Exercício 2) Agora em JAVA, crie uma classe denominada Elevador para armazenar as informações de um elevador dentro de um prédio.

A classe deve armazenar:
    Andar atual (0=térreo),
    total de andares no prédio, excluindo o térreo
    capacidade do elevador,
    quantas pessoas estão presentes nele.

A classe deve também disponibilizar os seguintes métodos:
    construtor: que deve receber como parâmetros: a capacidade do elevador e o total de andares no prédio
                     (os elevadores sempre começam no térreo e vazios);
    entra:       para acrescentar uma pessoa no elevador (só deve acrescentar se ainda houver espaço);
    sai:           para remover uma pessoa do elevador (só deve remover se houver alguém dentro dele);
    sobe:       para subir um andar (não deve subir se já estiver no último andar);
    desce:     para descer um andar (não deve descer se já estiver no térreo);
    get....:      métodos para obter cada um dos os dados armazenados.

Construa um menu completo no método Main para utilizar esse classe. Faça com que sua aplicação execute em um laço infinito, sendo que o encerramento deverá ser escolha do usuário.