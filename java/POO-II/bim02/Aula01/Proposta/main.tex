Como sabemos, a disciplina é dividida em duas etapas, a primeira com C++ e a segunda, agora, com Java.

Existem diferenças importantes entre essas linguagens, dessa forma, caberá a você realizar uma pesquisa e montar um documento Word ou Docs comparando essas diferenças, além de citar as semelhanças.

Alguns pontos(ou perguntas) importantes que podem lhe ajudar na sua pesquisa são:

- O que é IDE? Qual pode ser utilizada em Java? O que é necessário para rodarmos programas em Java na nossa máquina?
- Ambas as linguagens permitem implementações utilizando o paradigma orientado a objetos?
- Os conceitos de abstração, classe, objeto, encapsulamento, herança e polimorfismo são os mesmos para ambas as linguagens?
- Quais as diferenças entre as linguagens no contexto de herança?
- Mostre um exemplo de uma classe simples em Java
- Mostre como é instanciado um objeto dessa classe em Java
- Quais as principais diferenças entre as linguagens no contexto de memória (alocação dinâmica e desalocação)?
- O que são Interfaces em Java? Como uma classe com essa finalidade é implementada em C++ ?
- Como é definido uma classe Abstrata em Java? Essa ideia de classe existe em C++?