Revisão
Estrutura linear

PILHA
FILA        3 operações fundamentais (acesso inserção e remoção);
LISTA     

PILHA -> LIFO (LAST IN FIRST OUT);
FILA -> FIFO (FIRST IN FIRST OUT);
LISTA -> GENERICA (PEGA QUALQUER INFORMAÇAO E DEVOLVE QUALQUER INFORMAÇAO)

PILHA -> (push(p,x) (inserção), pop(p) (remoção),top(p) (acesso))
FILA -> (enqueue(f,x) (inserção),dequeue(f) (remoção),front(f) (acesso))
LISTA -> (insert(l,p,x) (inserção), remove(l,p) (remoção),element(l,p) (acesso),pos(l,x) (posição acesso) )

A partir da lista todas as estruturas lineares podem ser derivadas!

LISTA -> vira uma PILHA se a inserção,acesso e remoção for somente uma das entradas.
LISTA -> vira FILA se a inserção for feita no final e acesso e remoção no inicio.

*Operação não canonica muito comum pop(p,x) -> x recebe o valor do parametro removido