`#include <iostream>'
Essa biblioteca é utilizada para lidar com a entrada e saída padrão, 
permitindo a interação do usuário com o programa por meio do console.

`#include <vector>'
Essa biblioteca fornece uma implementação de vetor dinâmico, 
permitindo o armazenamento e manipulação de elementos em uma sequência dinamicamente
redimensionável.

`#include <queue>'
Essa biblioteca fornece uma implementação de fila (FIFO - First In, First Out), 
que é usada para implementar a busca em largura (BFS) no sistema metroviário.

`#include <stack>'
Essa biblioteca fornece uma implementação de pilha (LIFO - Last In, First Out), 
que é usada para implementar a busca em profundidade (DFS) no sistema metroviário.

`#include <climits>'
Essa biblioteca fornece constantes com os limites das variáveis numéricas, como INT_MAX e INT_MIN, 
que são usadas para inicializar os pesos das conexões do sistema metroviário.

`#include <algorithm>'
Essa biblioteca fornece funções genéricas para operações em intervalos de elementos, 
como find_if, que é usada para encontrar uma estação específica a ser removida.

`#include <limits>'
Essa biblioteca fornece constantes com os limites das variáveis numéricas, 
como numeric_limits<streamsize>::max(), 
que é usado para limpar o buffer de entrada quando uma entrada inválida é detectada.



`Função createGraph()'

1. Declaração e inicialização de variáveis:
A função começa declarando uma variável numStations para armazenar o número de estações do sistema.
Em seguida, exibe uma mensagem solicitando que o usuário digite o número de estações.
O loop while é usado para verificar se a entrada do usuário é um número inteiro. 
Caso contrário, uma mensagem de "entrada inválida" é exibida e a entrada é descartada até que um número inteiro válido seja fornecido pelo usuário.
Quando um número inteiro válido é inserido, ele é armazenado na variável numStations.

2. Redimensionamento das estruturas de dados:
Com base no número de estações fornecido pelo usuário, os vetores stations, adjacencyList e adjacencyMatrix são redimensionados para ter o tamanho correto.
O vetor stations armazenará as informações de cada estação, enquanto adjacencyList e adjacencyMatrix serão usados para representar as conexões entre as estações.

3. Captura dos nomes das estações:
Em um loop for, o programa solicita ao usuário o nome de cada estação e o armazena na estrutura de dados correspondente no vetor stations.

4. Captura das conexões entre as estações:
Em outro loop for, o programa solicita ao usuário o tempo necessário para ir de cada estação para todas as outras estações, exceto para si mesma.
O tempo necessário é armazenado na matriz de adjacência adjacencyMatrix e nas listas de adjacência adjacencyList para as estações correspondentes.
Cada conexão é representada por um par contendo o índice da estação de destino e o tempo necessário.

5. Mensagem de confirmação:
Após a conclusão dos loops e a captura de todas as informações, uma mensagem de "Grafo criado com sucesso!" é exibida.

Conclusao createGraph():
Essa função é responsável por configurar a estrutura de dados que representa o sistema metroviário com base nas informações fornecidas pelo usuário.
Ela cria um grafo direcionado onde cada estação é um nó e as conexões entre as estações são as arestas com os respectivos tempos de percurso.

`Função addStation()'

1. Captura do nome da nova estação:
A função começa solicitando ao usuário o nome da nova estação e armazena-o na variável name.

2.Criação da nova estação:
Uma variável newStation do tipo Station (que parece ser uma estrutura de dados definida anteriormente) 
é criada para representar a nova estação.
O nome da estação fornecido pelo usuário é atribuído à propriedade name da nova estação.

3. Captura das conexões da nova estação:
O programa solicita ao usuário o tempo necessário para ir da nova estação para cada uma das estações existentes.
Em um loop for, iterando sobre as estações existentes, exceto a estação recém-adicionada, o programa solicita ao usuário o tempo necessário para ir da nova estação para cada estação existente.
O tempo necessário é armazenado em um par, onde o primeiro elemento é o índice da estação existente e o segundo elemento é o tempo de percurso.
Esse par é adicionado à lista de conexões da nova estação (newStation.connections).
Além disso, o par também é adicionado à lista de adjacência (adjacencyList) e à matriz de adjacência (adjacencyMatrix).

4.Atualização das estruturas de dados:
A nova estação (newStation) é adicionada ao vetor stations.
A lista de conexões da nova estação é adicionada à lista de adjacência (adjacencyList).
Uma nova linha é adicionada à matriz de adjacência (adjacencyMatrix) para a nova estação, preenchida com valores padrão de -1 para indicar a falta de conexões.

5.Mensagem de confirmação:
Após a conclusão dos loops e a adição da nova estação e suas conexões às estruturas de dados, uma mensagem de "Estação adicionada com sucesso!" é exibida.

Conclusao addStation(): 
Essa função permite a expansão do sistema metroviário existente,
permitindo ao usuário adicionar novas estações e especificar suas conexões 
com as estações existentes.

`Função removeStation()'

1.Captura do nome da estação a ser removida:
A função solicita ao usuário o nome da estação que deseja remover e armazena-o na variável name.

2.Busca da estação na lista de estações existentes:
A função utiliza a função find_if para localizar a estação na lista stations com base no seu nome. A expressão lambda passada para find_if verifica se o nome da estação atual é igual ao nome fornecido pelo usuário.
Se a estação for encontrada, um iterador it apontando para a estação é retornado. Caso contrário, it será igual a stations.end().

3.Remoção da estação e suas conexões:
Se a estação for encontrada (it != stations.end()), o programa executa as seguintes etapas:
O índice da estação a ser removida é obtido usando a função distance, que calcula a distância entre stations.begin() e it.
A estação é removida do vetor stations usando a função erase.
A lista de adjacência (adjacencyList) e a matriz de adjacência (adjacencyMatrix) são atualizadas removendo as entradas correspondentes ao índice da estação removida.
Os loops for subsequentes percorrem a lista de adjacência (adjacencyList) e a matriz de adjacência (adjacencyMatrix) e removem as conexões que apontam para a estação removida.
Na lista de adjacência, a função remove_if é utilizada para remover os pares que têm o índice da estação removida como primeiro elemento.
Na matriz de adjacência, a coluna correspondente ao índice da estação removida é removida.
Por fim, uma mensagem de "Estação removida com sucesso!" é exibida.

4.Mensagem de estação não encontrada:
Se a estação não for encontrada (ou seja, it == stations.end()), é exibida a mensagem "Estação não encontrada!".

Conclusao removeStation():
Essa função permite ao usuário remover uma estação específica do sistema metroviário, 
atualizando as estruturas de dados (stations, adjacencyList e adjacencyMatrix) para refletir a remoção da estação e suas conexões. 

`Função breadthFirstSearch(const string &startStation)'

1.Busca da estação de partida na lista de estações existentes:
A função utiliza a função find_if para localizar a estação de partida na lista stations com base no seu nome. A expressão lambda passada para find_if verifica se o nome da estação atual é igual ao nome fornecido como argumento startStation.
Se a estação for encontrada, um iterador it apontando para a estação é retornado. Caso contrário, it será igual a stations.end().

2.Execução da busca em largura:
Se a estação de partida for encontrada (it != stations.end()), o programa executa as seguintes etapas:
O índice da estação de partida é obtido usando a função distance, que calcula a distância entre stations.begin() e it.
É criado um vetor visited de tamanho igual ao número de estações, inicializado com false. Esse vetor será usado para marcar as estações visitadas durante a busca.
É criada uma fila (queue) para armazenar os índices das estações a serem processadas.
Uma mensagem indicando a estação de partida é exibida.
A estação de partida é marcada como visitada (visited[startIndex] = true) e adicionada à fila (queue.push(startIndex)).

3.Realização da busca em largura:
Enquanto a fila não estiver vazia, o programa executa o seguinte loop:
O índice da estação atual é obtido a partir do elemento da frente da fila (int current = queue.front()).
O elemento da frente da fila é removido (queue.pop()).
O nome da estação atual é exibido (cout << stations[current].name << " ").
Para cada conexão da estação atual, representada por um par (connection) na lista de adjacência (adjacencyList[current]):
O índice da estação vizinha é obtido (int neighbor = connection.first).
Se a estação vizinha não tiver sido visitada (!visited[neighbor]), o programa executa as seguintes etapas:
A estação vizinha é marcada como visitada (visited[neighbor] = true).
O índice da estação vizinha é adicionado à fila (queue.push(neighbor)).

4.Exibição dos resultados:
Após concluir a busca em largura, uma nova linha é exibida (cout << endl).

5.Mensagem de estação não encontrada:
Se a estação de partida não for encontrada (ou seja, it == stations.end()), é exibida a mensagem "Estação não encontrada!".

Conclusao breadthFirstSearch(const string &startStation):
Essa função permite ao usuário realizar uma busca em largura a partir de uma estação específica no sistema metroviário.
Ela percorre as conexões das estações de forma ampla, explorando todas as estações alcançáveis a partir da estação de partida antes de avançar para as estações subsequentes.

`Função depthFirstSearch(const string &startStation)'

1.Busca da estação de partida na lista de estações existentes:
A função utiliza a função find_if para localizar a estação de partida na lista stations com base no seu nome. A expressão lambda passada para find_if verifica se o nome da estação atual é igual ao nome fornecido como argumento startStation.
Se a estação for encontrada, um iterador it apontando para a estação é retornado. Caso contrário, it será igual a stations.end().

2.Execução da busca em profundidade:
Se a estação de partida for encontrada (it != stations.end()), o programa executa as seguintes etapas:
O índice da estação de partida é obtido usando a função distance, que calcula a distância entre stations.begin() e it.
É criado um vetor visited de tamanho igual ao número de estações, inicializado com false. Esse vetor será usado para marcar as estações visitadas durante a busca.
É criada uma pilha (stack) para armazenar os índices das estações a serem processadas.
Uma mensagem indicando a estação de partida é exibida.
A estação de partida é marcada como visitada (visited[startIndex] = true) e adicionada à pilha (stack.push(startIndex)).

3.Realização da busca em profundidade:
Enquanto a pilha não estiver vazia, o programa executa o seguinte loop:
O índice da estação atual é obtido do topo da pilha (int current = stack.top()).
O elemento do topo da pilha é removido (stack.pop()).
O nome da estação atual é exibido (cout << stations[current].name << " ").
Para cada conexão da estação atual, representada por um par (connection) na lista de adjacência (adjacencyList[current]):
O índice da estação vizinha é obtido (int neighbor = connection.first).
Se a estação vizinha não tiver sido visitada (!visited[neighbor]), o programa executa as seguintes etapas:
A estação vizinha é marcada como visitada (visited[neighbor] = true).
O índice da estação vizinha é adicionado à pilha (stack.push(neighbor)).

4.Exibição dos resultados:
Após concluir a busca em profundidade, uma nova linha é exibida (cout << endl).

5.Mensagem de estação não encontrada:
Se a estação de partida não for encontrada (ou seja, it == stations.end()), é exibida a mensagem "Estação não encontrada!".

Conclusao depthFirstSearch(const string &startStation):
Essa função permite ao usuário realizar uma busca em profundidade a partir de uma estação específica no sistema metroviário. 
Ela explora o máximo possível em um ramo antes de retroceder e explorar os ramos subsequentes.

`Função findPath(const string &startStation, const string &endStation)'

1.Busca das estações de partida e destino na lista de estações existentes:
A função utiliza a função find_if para localizar a estação de partida e a estação de destino na lista stations com base em seus nomes. Duas expressões lambda são usadas para verificar se o nome da estação atual é igual aos nomes fornecidos como argumentos startStation e endStation.
Se as estações de partida e destino forem encontradas (startIt != stations.end() e endIt != stations.end()), os iteradores startIt e endIt apontando para as estações são retornados. Caso contrário, os iteradores serão iguais a stations.end().

2.Execução da busca em largura com rastreamento do caminho:
Se as estações de partida e destino forem encontradas, o programa executa as seguintes etapas:
Os índices das estações de partida e destino são obtidos usando a função distance, que calcula a distância entre stations.begin() e os iteradores startIt e endIt.
É criado um vetor path para armazenar o caminho percorrido. Esse vetor inicialmente contém apenas o índice da estação de partida (startIndex).
É criado um vetor visited de tamanho igual ao número de estações, inicializado com false. Esse vetor será usado para marcar as estações visitadas durante a busca.
É criada uma fila (queue) para armazenar pares de índices de estações e os caminhos percorridos até o momento.
A estação de partida é marcada como visitada (visited[startIndex] = true) e adicionada à fila com seu respectivo caminho (queue.push(make_pair(startIndex, vector<int>{startIndex}))).

3.Realização da busca em largura com rastreamento do caminho:
Enquanto a fila não estiver vazia, o programa executa o seguinte loop:
O índice da estação atual é obtido da frente da fila (int current = queue.front().first).
O caminho percorrido até o momento é atualizado com o caminho armazenado na frente da fila (path = queue.front().second).
O elemento da frente da fila é removido (queue.pop()).
Se a estação atual for igual à estação de destino (current == endIndex), isso significa que foi encontrado um caminho entre as duas estações:
Uma mensagem indicando as estações de partida e destino é exibida (cout << "Caminho encontrado entre as estações " << startStation << " e " << endStation << ":" << endl).
Para cada índice de estação no caminho, o nome da estação correspondente é exibido (cout << stations[stationIndex].name << " ").
Uma nova linha é exibida (cout << endl).
A função retorna, encerrando sua execução.
Caso contrário, se a estação atual não for a estação de destino, o programa executa o seguinte:
Para cada conexão da estação atual, representada por um par (connection) na lista de adjacência (adjacencyList[current]):
O índice da estação vizinha é obtido (int neighbor = connection.first).
Se a estação vizinha não tiver sido visitada (!visited[neighbor]), o programa executa as seguintes etapas:
A estação vizinha é marcada como visitada (visited[neighbor] = true).
É criado um novo vetor newPath com base no caminho atual (vector<int> newPath = path).
O índice da estação vizinha é adicionado ao novo caminho (newPath.push_back(neighbor)).
O par contendo o índice da estação vizinha e o novo caminho é adicionado à fila (queue.push(make_pair(neighbor, newPath))).

4.Exibição dos resultados:
Se não for possível encontrar um caminho entre as estações de partida e destino, uma mensagem é exibida informando isso (cout << "Não foi possível encontrar um caminho entre as estações " << startStation << " e " << endStation << endl).

5.Mensagem de estação não encontrada:
Se a estação de partida ou a estação de destino não forem encontradas (ou seja, startIt == stations.end() ou endIt == stations.end()), é exibida a mensagem "Estação não encontrada!".

Conclusao findPath(const string &startStation, const string &endStation):
Essa função permite ao usuário encontrar um caminho entre duas estações no sistema metroviário usando uma busca em largura com rastreamento do caminho percorrido. 
Ela exibe o caminho encontrado, caso exista, ou uma mensagem informando que não foi possível encontrar um caminho entre as duas estações.

`Função findShortestPath(const string &startStation, const string &endStation)'

1.Busca das estações de partida e destino na lista de estações existentes:
A função utiliza a função find_if para localizar a estação de partida e a estação de destino na lista stations com base em seus nomes. Duas expressões lambda são usadas para verificar se o nome da estação atual é igual aos nomes fornecidos como argumentos startStation e endStation.
Se as estações de partida e destino forem encontradas (startIt != stations.end() e endIt != stations.end()), os iteradores startIt e endIt apontando para as estações são retornados. Caso contrário, os iteradores serão iguais a stations.end().

2.Execução do algoritmo de Dijkstra para encontrar o caminho mais curto:
Se as estações de partida e destino forem encontradas, o programa executa as seguintes etapas:
Os índices das estações de partida e destino são obtidos usando a função distance, que calcula a distância entre stations.begin() e os iteradores startIt e endIt.
É criado um vetor distances para armazenar as distâncias mais curtas a partir da estação de partida até cada estação do sistema. Todas as distâncias são inicializadas como INT_MAX para indicar que ainda não foram calculadas.
É criado um vetor previous para armazenar os predecessores de cada estação no caminho mais curto. Inicialmente, todos os predecessores são definidos como -1.
É criado um vetor visited para marcar as estações visitadas durante o algoritmo. Inicialmente, todas as estações são marcadas como não visitadas (false).
É criada uma fila de prioridade (pq) para armazenar pares de distâncias e índices de estações. A fila é ordenada de forma crescente com base nas distâncias.
A distância da estação de partida para ela mesma é definida como 0 (distances[startIndex] = 0).
O par contendo a distância 0 e o índice da estação de partida é inserido na fila de prioridade (pq.push(make_pair(0, startIndex))).

3.Execução do loop principal do algoritmo de Dijkstra:
Enquanto a fila de prioridade não estiver vazia, o programa executa o seguinte loop:
O índice da estação atual é obtido a partir do topo da fila de prioridade (int current = pq.top().second).
O elemento do topo da fila de prioridade é removido (pq.pop()).
Se a estação atual for igual à estação de destino (current == endIndex), isso significa que o caminho mais curto foi encontrado:
O loop é interrompido usando o comando break.
Se a estação atual já tiver sido visitada (visited[current]), o loop continua para a próxima iteração.
A estação atual é marcada como visitada (visited[current] = true).
Para cada conexão da estação atual, representada por um par (connection) na lista de adjacência (adjacencyList[current]):
O índice da estação vizinha é obtido (int neighbor = connection.first).
O peso da conexão é obtido (int weight = connection.second).
É calculada uma nova distância até a estação vizinha (int newDistance = distances[current] + weight).
Se a nova distância for menor que a distância atualmente armazenada para a estação vizinha (newDistance < distances[neighbor]):
A distância é atualizada com o novo valor (distances[neighbor] = newDistance).
O predecessor da estação vizinha é atualizado como a estação atual (previous[neighbor] = current).
O par contendo a nova distância e o índice da estação vizinha é inserido na fila de prioridade (pq.push(make_pair(newDistance, neighbor))).

4.Exibição do resultado:
Se a distância até a estação de destino for diferente de INT_MAX, isso significa que um caminho foi encontrado:
É exibida a mensagem "Menor caminho encontrado entre as estações [nome da estação de partida] e [nome da estação de destino]:".
É criada uma pilha pathStack para armazenar os índices das estações no caminho mais curto.
O índice da estação de destino é atribuído à variável current.
Enquanto current for diferente de -1 (enquanto houver predecessores no caminho):
O índice da estação é inserido na pilha pathStack (pathStack.push(current)).
O predecessor de current é atribuído à variável current (current = previous[current]).
Enquanto a pilha pathStack não estiver vazia:
O índice da estação é retirado do topo da pilha (int stationIndex = pathStack.top()).
O índice é removido da pilha (pathStack.pop()).
O nome da estação correspondente ao índice é exibido (cout << stations[stationIndex].name << " ").
É exibido um novo espaço em branco (cout << endl).
É exibido o tempo total do caminho (cout << "Tempo do caminho: " << distances[endIndex] << endl).
Caso contrário, é exibida a mensagem "Não foi possível encontrar um caminho entre as estações [nome da estação de partida] e [nome da estação de destino]".

5.Mensagem de estação não encontrada:
Se a estação de partida ou a estação de destino não forem encontradas (ou seja, startIt == stations.end() ou endIt == stations.end()), é exibida a mensagem "Estação não encontrada!".

Conclusao findShortestPath(const string &startStation, const string &endStation):
Essa função permite ao usuário encontrar o caminho mais curto (com base na menor distância) entre duas estações no sistema metroviário,
levando em consideração os pesos das conexões entre as estações. 
Além disso, ela exibe o caminho encontrado, caso exista, ou uma mensagem informando que não foi possível encontrar um caminho entre as estações especificadas.


`Função findMinimumSpanningTree()'

1.Inicialização:
Vetor visited: Armazena o estado de visita das estações. Inicialmente, todas as estações são marcadas como não visitadas (false).
Vetor distances: Armazena as distâncias mínimas entre as estações e a Árvore Geradora Mínima. Inicialmente, todas as distâncias são definidas como infinito (INT_MAX).
Vetor parent: Armazena os pais de cada estação na Árvore Geradora Mínima. Inicialmente, todos os pais são definidos como -1, indicando que não possuem pai.
A distância da estação inicial (índice 0) é definida como 0 (distances[0] = 0).

2.Etapa principal:
O loop externo é executado stations.size() - 1 vezes (uma vez para cada estação, exceto a inicial).
A cada iteração, encontra-se a estação mais próxima que ainda não foi visitada:
minDistance: Armazena a menor distância encontrada até o momento, inicialmente definida como infinito (INT_MAX).
minIndex: Armazena o índice da estação com a menor distância encontrada, inicialmente definido como -1.

3.O loop interno percorre todas as estações:
Verifica se a estação não foi visitada e se a distância até ela é menor que a menor distância atual (!visited[j] && distances[j] < minDistance).
Se a condição for verdadeira, atualiza minDistance com a nova menor distância encontrada e minIndex com o índice da estação.
Após encontrar a estação mais próxima, marca-a como visitada (visited[minIndex] = true).
Atualiza as distâncias mínimas e os pais das estações vizinhas à estação atual:
Para cada conexão da estação atual (adjacencyList[minIndex]):
Obtém o índice da estação vizinha (int neighbor = connection.first).
Obtém o peso da conexão entre as estações (int weight = connection.second).
Verifica se a estação vizinha não foi visitada e se o peso da conexão é menor que a distância mínima atual até a estação vizinha (!visited[neighbor] && weight < distances[neighbor]).
Se a condição for verdadeira, atualiza o pai da estação vizinha como a estação atual (parent[neighbor] = minIndex) e atualiza a distância mínima até a estação vizinha com o peso da conexão (distances[neighbor] = weight).

4.Exibição da Árvore Geradora Mínima:
Após o loop, exibe-se a mensagem "Árvore geradora mínima do sistema metroviário:".
Em seguida, percorre-se as estações a partir do índice 1 (excluindo a estação inicial) e exibe-se o nome da estação pai (stations[parent[i]].name) seguido do nome da estação atual (stations[i].name), indicando uma conexão na Árvore Geradora Mínima.

Conclusao findMinimumSpanningTree():
Essa função implementa o algoritmo de Prim para encontrar a Árvore Geradora Mínima (Minimum Spanning Tree) no sistema metroviário.
Dessa forma, a função encontra e exibe a Árvore Geradora Mínima do sistema metroviário com base nas conexões e pesos das estações.

`Função printAdjacencyList()'

1.Imprime a mensagem "Lista de adjacência do sistema metroviário:" 
para indicar a saída que será exibida.

2.Inicia um loop que percorre cada estação no vetor stations.

3.Para cada estação: 
Imprime o nome da estação atual (stations[i].name) seguido de uma seta (->).

4.Em seguida, percorre as conexões da estação atual no vetor adjacencyList[i].

5.Para cada conexão:
Imprime o nome da estação vizinha (stations[connection.first].name) seguido do peso da conexão (connection.second).

6.Após percorrer todas as conexões da estação atual, imprime uma quebra de linha (endl).

Conclusao printAdjacencyList():
Dessa forma, a função percorre a lista de adjacência do sistema metroviário 
e imprime as conexões entre as estações, juntamente com os pesos das conexões. 
Isso permite visualizar de forma clara as relações de conectividade no sistema metroviário.

`Função printAdjacencyMatrix()'

1.Imprime a mensagem "Matriz de adjacência do sistema metroviário:" 
Para indicar a saída que será exibida.

2.Inicia dois loops aninhados:
Sendo o primeiro responsável por percorrer as linhas da matriz (estações de partida) e o segundo por percorrer as colunas (estações de chegada).

3.Para cada posição da matriz, verifica se o valor armazenado em adjacencyMatrix[i][j] é diferente de -1. Isso indica a existência de uma conexão entre as estações i e j.

4.Se o valor for diferente de -1, significa que há uma conexão e imprime o peso da conexão (adjacencyMatrix[i][j]) seguido de um caractere de tabulação (\t).

5.Caso o valor seja igual a -1, significa que não há conexão direta entre as estações i e j, e imprime um traço (-) seguido de um caractere de tabulação.

6.Após percorrer todas as colunas da linha atual, imprime uma quebra de linha (endl) para passar para a próxima linha da matriz.

Conclusao printAdjacencyMatrix():
Dessa forma, a função percorre a matriz de adjacência do sistema metroviário 
e imprime os pesos das conexões entre as estações. 
Isso permite visualizar de forma tabular as conexões existentes e seus respectivos pesos, 
facilitando a análise e compreensão da estrutura do sistema metroviário.

`CODIGO'
O código em questão implementa um sistema metroviário utilizando conceitos de grafos. 
Ele permite adicionar estações, criar conexões entre elas com tempos de percurso, e realiza operações como busca em largura, busca em profundidade, busca de caminho mínimo, criação da árvore geradora mínima e exibição da lista de adjacência e matriz de adjacência.
Em resumo, o código permite construir e manipular um sistema metroviário, realizando operações como busca de caminhos, determinação do caminho mais curto, exibição das conexões entre as estações e criação da árvore geradora mínima. É uma implementação básica de um sistema metroviário usando estruturas de grafos.