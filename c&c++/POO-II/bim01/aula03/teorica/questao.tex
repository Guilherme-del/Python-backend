Baseado nos conceitos que vimos na última aula e no exemplo implementado do "Elevador", implemente uma abstração de um carro. Abaixo estão as especificações.

-  O tanque de combustível do carro armazena no máximo 50 litros de gasolina.
-  O carro consome 15 km/litro.
   
-  Deve ser possível realizar as seguintes ações:
        Abastecer o carro com uma certa quantidade de gasolina;
        Mover o carro em uma determinada distância (medida em km);
        Retornar a quantidade de combustível e a distância total percorrida.
   
-  No programa principal, crie 2 carros e realize os seguintes passos:
        Abasteça 20 litros no primeiro e 30 litros no segundo.
        Desloque o primeiro em 200 km e o segundo em 400 km.
        Exiba na tela a distância percorrida e o total de combustível restante para cada um.